%--- routing ------------------------------------------------------------------------------
\section{Routing}
%--- routing------------------------------------------------------------------------------
\begin{frame}[fragile] \frametitle{Routing}
\begin{itemize}
  \item the browser history is part of the user experience
  \item allows the user to navigate back to earlier visited pages
  \item an entry in the history is added when the user
  \begin{itemize}
    \item navigates to a new page using a link
    \item submits a form
  \end{itemize}
  \item traditionally, this loads a new page from the server
  \item when a new page is loaded, all JavaScript objects are lost
  \item singel page web application prevents this using \code{preventDefault()} on all relevant events
  \item only updating the DOM will impact the user experience:
  \begin{itemize}
    \item can not navigate using the browser history (back button)
    \item can not link to inner pages
  \end{itemize}
\end{itemize}
\end{frame}

%--- Routing Framework------------------------------------------------------------------------------
\begin{frame}[fragile] \frametitle{Routing Framework}
\begin{itemize}
  \item there is an API giving JavaScript direct access to the browser history
  \item using it manually is tedious and error prone
  \item let a router do the work for you
  \begin{itemize}
    \item subscribing and manipulating the history stack
    \item matching the URL to your routes
    \item rendering a nested UI from the route matches
  \end{itemize}
\end{itemize}

\vspace{10mm}
\code{npm install react-router-dom}
\end{frame}

%--- Link ------------------------------------------------------------------------------
\section{Link}
%--- Link ------------------------------------------------------------------------------
\begin{frame}[fragile] \frametitle{Link}
\code{<Link to="/animals">animals</Link>}
\\ \code{<Link to="animals/fish">fishs</Link>}
\begin{itemize}
    \item a react component
    \item let users navigate in your app
    \item clicking on it will add an entry to the browser history
    \item page is not fetched (\code{preventDefaults} on \code<a href="...">)
    \item this will update the url field in the browser
    \begin{itemize}
      \item \code{<Route>} triggers re-render
    \end{itemize}
    \item your JavaScript objects are untouched (preserve the application state)
\end{itemize}
\end{frame}

%--- NavLink ------------------------------------------------------------------------------
\begin{frame}[fragile] \frametitle{NavLink}
Add styling of active link using:
\begin{itemize}
  \item knows if it is "active" or "pending"
  \item use css class to highlight active links
  \item \code{className} -- normal CSS, or a function returning the css class
  \item by default, an \code{active} CSS class is added when active
\end{itemize}

\begin{CodeBox}{}
<NavLink
  to="/messages"
  className={({ isActive, isPending }) =>
    isPending ? "pending" : isActive ? "active" : ""
  }
>\end{CodeBox}
\end{frame}

%--- Link Example------------------------------------------------------------------------------
\begin{frame}[fragile] \frametitle{Link Example}
\begin{CodeBox}{}
import { Link } from 'react-router-dom';
function Menu() {
  return (
    <nav>
      <Link to="animal" />
      <Link to="animal/fish" />
      <Link to="animal/bird" />
    </nav>
  );
}
\end{CodeBox}
\end{frame}

%--- Route ------------------------------------------------------------------------------
\section{Route}
%--- Route ------------------------------------------------------------------------------
\begin{frame}[fragile] \frametitle{createBrowserRouter}
\begin{itemize}
  \item renders components based on url matching
  \item routes are declared in a configuration object
  \item add to your app using the \code{<RouterProvider>} component
\end{itemize}
\vspace{5mm}
\begin{CodeBox}{}
const router = createBrowserRouter(config);

ReactDOM.createRoot(document.getElementById("root")).render(
  <React.StrictMode>
    <RouterProvider router={router} />
  </React.StrictMode>
);
\end{CodeBox}
\end{frame}

%--- Route objects ------------------------------------------------------------------------------
\begin{frame}[fragile] \frametitle{Route objects}
\code{router} is an array of Route objects

properties:
\begin{itemize}
  \item \code{path} -- string for match against the url
  \item \code{element/Component}  -- rendered when path is matched
  \begin{itemize}
    \item \code{element} -- an object (JSX expression)
    \item \code{Component} -- a react component (JavaScript function)
    \item use one of them
  \end{itemize}
  \item \code{caseSensitive}  -- if pattern matching is case sensitive (default fale)
  \item \code{children} -- nested routes
  \item error handling
  \begin{itemize}
    \item \code{errorElement} -- an object (JSX expression)
    \item \code{ErrorBoundary} -- a react component (JavaScript function)
  \end{itemize}
  \item and more, covered later
\end{itemize}
\end{frame}

%--- Route ------------------------------------------------------------------------------
\begin{frame}[fragile] \frametitle{createBrowserRouter}
\begin{CodeBox}{}
const router = createBrowserRouter([
  {
    path: "/",
    element: <h1>Welcome</h1>,
  }, {
    path: 'hello/world',
    caseSensitive: true,
    element: <h1>Hello World</h1>
  }, {
    path: 'about/*',
    Component: {WildcardComponent}
  },
]);

\end{CodeBox}
\end{frame}

%--- Path pattern ------------------------------------------------------------------------------
\begin{frame}[fragile] \frametitle{Path}
\begin{itemize}
  \item only one match in the array
  \item most constrained wins, order do not matter
  \item a path is composed of segments: pattern between '/'
  \item text segment
  \begin{itemize}
    \item exact match, letter by letter
    \item case sensitivity is optional
    \item percent decoded text, do not use \code{\%nn}
    \item use url safe characters
    \item path: \texttt{/kebab-case/in-path/next-segment}
    \item matches url: \texttt{kebab-case/in-path/next-segment}
  \end{itemize}
\end{itemize}
\end{frame}

%--- segments ------------------------------------------------------------------------------
\begin{frame}[fragile] \frametitle{Segments}
\begin{itemize}
  \item dynamic segment: starts with ':'
  \begin{itemize}
    \item matches any characters, zero or more
    \item the matched url text can be accessed in your component
    \item path: \code{user/:id/lang/:lang}
    \item matches url: \code{user/31/lang/se}
  \end{itemize}
  \item optional segment, ends with '?'
  \begin{itemize}
    \item path: \code{:lang?/categories}
    \item matchs url: \code{categories}, and \code{en/categories}
  \end{itemize}
  \item splats, catchall, star: ends with '/*', matches any character following the /, including other /
  \begin{itemize}
    \item path: \code{files/*}
    \item matchs url: \code{files/one/two/three}
  \end{itemize}
\end{itemize}
\end{frame}

%--- Layout and index ------------------------------------------------------------------------------
\begin{frame}[fragile] \frametitle{Layout and Index routes}
\begin{itemize}
  \item Layout route
  \begin{itemize}
    \item do not have a \code{path}
    \item should not have siblings
    \item the element/Component is always rendered, adds to the layout
    \item do not consume url segments
  \end{itemize}
  \item Index routes
  \begin{itemize}
    \item selected if parent is matched but no siblings
    \item do not have a \code{path}
    \item matches an empty url segment
    \item do have a \code{index} attribute
  \end{itemize}
\end{itemize}
\end{frame}

%--- Nested Routs ------------------------------------------------------------------------------
\begin{frame}[fragile] \frametitle{Nested Routs}
Routs can be nested:
\begin{itemize}
  \item \code{children} attribute of a Route object -- an array of child routs
  \item each level matches a part of the url
  \item at most one \code{path} in the child array will be matched
  \item the element/Component of the matched \code{path} will be rendered on each level
\end{itemize}
\end{frame}


%--- Nested Routs ------------------------------------------------------------------------------
\begin{frame}[fragile] \frametitle{Nested Routs}
\begin{CodeBox}{}
const routerConfig= [{
        path: "shop",
        element: <ShopFrame />,
        children: [
          {
            path: "item/:id",
            element: <Item />,
          }, {
            index: true,
            element: <ListItems />
          }]
      }, {
        path: "admin",
        element: <AdminFrame />,
        children: [
          {
            path: "edit-item/:id",
            element: <EditItem />,
          }, {
            index: true,
            element: <EditListItems />
          }]
      }]
\end{CodeBox}
\end{frame}

%--- Outlet ------------------------------------------------------------------------------
\begin{frame}[fragile] \frametitle{\code{Outlet}}
The child element is rendered in the parents \code{Outlet}

\vspace{5mm}
\begin{CodeBox}{}
import { Outlet } from 'react-router-dom';

function ShopFrame() {
  return (
    <div>
      <h1>The Shop</h1>
      <ShopMenu />
      <Outlet />
    </div>
  );
}
\end{CodeBox}
\end{frame}

%--- Outlet, pass props ------------------------------------------------------------------------------
\begin{frame}[fragile] \frametitle{\code{Outlet} --- passing props}
Use \code{useOutletContext()} to get data in child component
\begin{CodeBox}{}
function ShopFrame() {
  const data = useState();
  return (
    <div>
      <Outlet context = data/>
    </div>
  );
}

function ChildComponent() {
    const { data } = useOutletContext();
    return <p>{data}</p>
}
\end{CodeBox}
\end{frame}

%--- Path Parameters ------------------------------------------------------------------------------
\begin{frame}[fragile] \frametitle{Path Parameters}
the router pass data from the path to the component
\begin{itemize}
  \item specify parameters in the \code{path} using the syntax\code{:name}
  \item use the \code{useParams()}  hook to get an object with the values
\end{itemize}

\vspace{5mm}
\begin{CodeBox}{}
import { useParams } from "react-router-dom";

const config = {
    path="/item/:itemId", 
    element={<Item />}
};

function Item() {
  let params = useParams();
  return <h2>item: {params.itemId}</h2>;
}
\end{CodeBox}
\end{frame}

%--- Hooks ------------------------------------------------------------------------------
\begin{frame}[fragile] \frametitle{Hooks}
\begin{itemize}
  \item can be used in any child of \code{Route}
  \item \code{useParams()} - returns an object with the URL path parameters
  \item \code{useLocation()} - returns the browser location
  \item \code{useSearchParams()} - interact with query string in the URL
  \item \code{useNavigate()} - navigate programatically
\end{itemize}
\end{frame}

%--- Error handling ------------------------------------------------------------------------------
\begin{frame}[fragile] \frametitle{Error handling}
When exceptions are thrown in loaders, actions, or component rendering:
\begin{itemize}
  \item the \code{element/Component} is not rendered
  \item instead the \code{errorElement/ErrorBoundary} is
  \begin{itemize}
    \item \code{errorElement} -- an object (JSX expression)
    \item \code{ErrorBoundary} -- a react component (JavaScript function)
    \item use one of them
  \end{itemize}
  \item exceptions will bubble up the router tree 
\end{itemize}
\end{frame}
%--- Error handling ------------------------------------------------------------------------------
\begin{frame}[fragile] \frametitle{Error handling}
\begin{CodeBox}{}
config = [{
  path: "/invoices/:id",
  element: <Invoice />,
  errorElement: <ErrorBoundary />
}];
function ErrorBoundary() {
  let error = useRouteError();
  console.error(error);
  return <div>Ooops! {error}</div>;
}
\end{CodeBox}
\end{frame}



%--- Picking a Router ------------------------------------------------------------------------------
\begin{frame}[fragile] \frametitle{Picking a Router}

For all urls belonging to the app:
\begin{itemize}
  \item the server must return the html file bootstrapping react, \code{index.html} 
\end{itemize}

\code{createBrowserRouter}
\begin{itemize}
  \item \code{http://domain.se/item/42}
  \item node.js built in server do this for you
  \item configure apache with rewrites
\end{itemize}

\code{createHashRouter}
\begin{itemize}
  \item \code{http://domain.se/#/item/42}
  \item compatible with all servers
\end{itemize}
\end{frame}

%--- React Router 6.4 example  ------------------------------------------------------------------------------
\begin{frame}[fragile] \frametitle{React Router pre 6.4 Example}
\vspace{-3mm}
\begin{CodeBox}{}
import { Route, Routes } from 'react-router-dom';
function App() {
  return (
    <Routes>
      <Route path="animal" element={<Animal />}>
        <Route path="fish"} element={<Fish />}/>
        <Route path="bird"} element={<Bird />}/>
        <Route index element={<SelectAnimal />}/>
      </Route>
    </Routes>
  );
}
\end{CodeBox}
\vspace{-5mm}
/animal/fish $\rightarrow$ \code{<Animal><Fish /><Animal>}
\\/animal/cat $\rightarrow$ no match
\end{frame}


%--- Data API ------------------------------------------------------------------------------
\begin{frame}[fragile] \frametitle{Data API}

We will return to react router data API:
\begin{itemize}
  \item actions
  \item loaders
  \item lazy
\end{itemize}
\vspace{5mm}
Based on \code{async} functions, so we need to cover that first.
\end{frame}
